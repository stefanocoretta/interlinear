% Options for packages loaded elsewhere
% Options for packages loaded elsewhere
\PassOptionsToPackage{unicode}{hyperref}
\PassOptionsToPackage{hyphens}{url}
\PassOptionsToPackage{dvipsnames,svgnames,x11names}{xcolor}
%
\documentclass[
  letterpaper,
  DIV=11,
  numbers=noendperiod]{scrartcl}
\usepackage{xcolor}
\usepackage{amsmath,amssymb}
\setcounter{secnumdepth}{-\maxdimen} % remove section numbering
\usepackage{iftex}
\ifPDFTeX
  \usepackage[T1]{fontenc}
  \usepackage[utf8]{inputenc}
  \usepackage{textcomp} % provide euro and other symbols
\else % if luatex or xetex
  \usepackage{unicode-math} % this also loads fontspec
  \defaultfontfeatures{Scale=MatchLowercase}
  \defaultfontfeatures[\rmfamily]{Ligatures=TeX,Scale=1}
\fi
\usepackage{lmodern}
\ifPDFTeX\else
  % xetex/luatex font selection
\fi
% Use upquote if available, for straight quotes in verbatim environments
\IfFileExists{upquote.sty}{\usepackage{upquote}}{}
\IfFileExists{microtype.sty}{% use microtype if available
  \usepackage[]{microtype}
  \UseMicrotypeSet[protrusion]{basicmath} % disable protrusion for tt fonts
}{}
\makeatletter
\@ifundefined{KOMAClassName}{% if non-KOMA class
  \IfFileExists{parskip.sty}{%
    \usepackage{parskip}
  }{% else
    \setlength{\parindent}{0pt}
    \setlength{\parskip}{6pt plus 2pt minus 1pt}}
}{% if KOMA class
  \KOMAoptions{parskip=half}}
\makeatother
% Make \paragraph and \subparagraph free-standing
\makeatletter
\ifx\paragraph\undefined\else
  \let\oldparagraph\paragraph
  \renewcommand{\paragraph}{
    \@ifstar
      \xxxParagraphStar
      \xxxParagraphNoStar
  }
  \newcommand{\xxxParagraphStar}[1]{\oldparagraph*{#1}\mbox{}}
  \newcommand{\xxxParagraphNoStar}[1]{\oldparagraph{#1}\mbox{}}
\fi
\ifx\subparagraph\undefined\else
  \let\oldsubparagraph\subparagraph
  \renewcommand{\subparagraph}{
    \@ifstar
      \xxxSubParagraphStar
      \xxxSubParagraphNoStar
  }
  \newcommand{\xxxSubParagraphStar}[1]{\oldsubparagraph*{#1}\mbox{}}
  \newcommand{\xxxSubParagraphNoStar}[1]{\oldsubparagraph{#1}\mbox{}}
\fi
\makeatother


\usepackage{longtable,booktabs,array}
\usepackage{calc} % for calculating minipage widths
% Correct order of tables after \paragraph or \subparagraph
\usepackage{etoolbox}
\makeatletter
\patchcmd\longtable{\par}{\if@noskipsec\mbox{}\fi\par}{}{}
\makeatother
% Allow footnotes in longtable head/foot
\IfFileExists{footnotehyper.sty}{\usepackage{footnotehyper}}{\usepackage{footnote}}
\makesavenoteenv{longtable}
\usepackage{graphicx}
\makeatletter
\newsavebox\pandoc@box
\newcommand*\pandocbounded[1]{% scales image to fit in text height/width
  \sbox\pandoc@box{#1}%
  \Gscale@div\@tempa{\textheight}{\dimexpr\ht\pandoc@box+\dp\pandoc@box\relax}%
  \Gscale@div\@tempb{\linewidth}{\wd\pandoc@box}%
  \ifdim\@tempb\p@<\@tempa\p@\let\@tempa\@tempb\fi% select the smaller of both
  \ifdim\@tempa\p@<\p@\scalebox{\@tempa}{\usebox\pandoc@box}%
  \else\usebox{\pandoc@box}%
  \fi%
}
% Set default figure placement to htbp
\def\fps@figure{htbp}
\makeatother





\setlength{\emergencystretch}{3em} % prevent overfull lines

\providecommand{\tightlist}{%
  \setlength{\itemsep}{0pt}\setlength{\parskip}{0pt}}



 


\KOMAoption{captions}{tableheading}
% Fix compatibility with unicode-math (https://github.com/wspr/unicode-math/issues/379#issuecomment-276079476)
\usepackage{expex}
\let\expexgla\gla
\AtBeginDocument{%
    \let\umgla\gla
    \let\gla\expexgla
}
    
\makeatletter
\@ifpackageloaded{caption}{}{\usepackage{caption}}
\AtBeginDocument{%
\ifdefined\contentsname
  \renewcommand*\contentsname{Table of contents}
\else
  \newcommand\contentsname{Table of contents}
\fi
\ifdefined\listfigurename
  \renewcommand*\listfigurename{List of Figures}
\else
  \newcommand\listfigurename{List of Figures}
\fi
\ifdefined\listtablename
  \renewcommand*\listtablename{List of Tables}
\else
  \newcommand\listtablename{List of Tables}
\fi
\ifdefined\figurename
  \renewcommand*\figurename{Figure}
\else
  \newcommand\figurename{Figure}
\fi
\ifdefined\tablename
  \renewcommand*\tablename{Table}
\else
  \newcommand\tablename{Table}
\fi
}
\@ifpackageloaded{float}{}{\usepackage{float}}
\floatstyle{ruled}
\@ifundefined{c@chapter}{\newfloat{codelisting}{h}{lop}}{\newfloat{codelisting}{h}{lop}[chapter]}
\floatname{codelisting}{Listing}
\newcommand*\listoflistings{\listof{codelisting}{List of Listings}}
\makeatother
\makeatletter
\makeatother
\makeatletter
\@ifpackageloaded{caption}{}{\usepackage{caption}}
\@ifpackageloaded{subcaption}{}{\usepackage{subcaption}}
\makeatother
\usepackage{bookmark}
\IfFileExists{xurl.sty}{\usepackage{xurl}}{} % add URL line breaks if available
\urlstyle{same}
\hypersetup{
  pdftitle={interlinear examples},
  colorlinks=true,
  linkcolor={blue},
  filecolor={Maroon},
  citecolor={Blue},
  urlcolor={Blue},
  pdfcreator={LaTeX via pandoc}}


\title{\texttt{interlinear} examples}
\author{}
\date{}
\begin{document}
\maketitle


The Quarto filter \texttt{interlinear} enables users to write numbered
linguistic examples. The filter can also render interlinear glosses. A
custom implementation for example numbering has been developed for the
HTML output, while in LaTeX it is handled by expex. Interlinear glossing
is handled by Leipzig.js in HTML and expex in LaTeX.

\subsection{Numbered examples}\label{numbered-examples}

To create a numbered example, use a fenced Div block (\texttt{:::}) with
class \texttt{ex}. For example the following code results in Example
(1).

\begin{verbatim}
::: ex
You do the job. You are not the job.
:::
\end{verbatim}

\pex
You do the job. You are not the job.
\xe

\subsection{Subexamples}\label{subexamples}

You can create subexamples within an \texttt{ex} Div block with a fenced
Div block with class \texttt{exi} (for ``example item'') for each
subexample.

\begin{verbatim}
::: ex
::: exi
A bird sings a beautiful song.
:::

::: exi
A fish swims in the pool.
:::
:::
\end{verbatim}

Which renders as:

\pex
\a A bird sings a beautiful song.

\a A fish swims in the pool.
\xe

\subsection{Interlinear glosses}\label{interlinear-glosses}

To create an interlinear gloss, use a fenced Div block (\texttt{:::})
with class \texttt{gl} (usually, within an \texttt{ex} Div). These are
the current rules:

\begin{itemize}
\tightlist
\item
  Each line of the interlinear gloss should start with
  \texttt{\textbar{}}.
\item
  Interlinear lines that start with \texttt{\textbar{}\ -} are treated
  as preambles/original text (i.e.~they are not aligned).
\item
  The last interlinear line is treated as the free translation.
\end{itemize}

For example:

\begin{verbatim}
::: ex
::: gl
| - Gila aburun ferma hamišaluǧ güǧüna amuq’dač.
| gila abur-u-n ferma hamišaluǧ güǧüna amuq’-da-č.
| now they-OBL-GEN farm forever behind stay-FUT-NEG
| 'Now their farm will not stay behind forever.'
:::
:::
\end{verbatim}

\phantomsection\label{ex-lez}
\pex
\begingl
\glpreamble  Gila aburun ferma hamišaluǧ güǧüna amuq’dač. //
\gla gila abur-u-n ferma hamišaluǧ güǧüna amuq’-da-č. //
\glb now they-OBL-GEN farm forever behind stay-FUT-NEG //
\glft ‘Now their farm will not stay behind forever.’ //
\endgl
\xe

To create a gloss without free translation, add a final empty line
starting with \texttt{\textbar{}\ -} (the following example is also
without the original text line):

\begin{verbatim}
::: ex
::: gl
| el perrito está comiendo
| the {little dog} is eating
| -
:::
:::
\end{verbatim}

Renders as:

\phantomsection\label{ex-sp}
\pex
\begingl
\gla el perrito está comiendo //
\glb the {little dog} is eating //
\endgl
\xe

You can include interlinear glosses in subexamples and mix glosses with
simple sentences. Note that more than three colons \texttt{:} can be
used for fenced Div blocks to show the different levels of embedding
within the \texttt{ex} Div (this is entirely optional).

\begin{verbatim}
::::: ex
::: exi
Abandon all hope, ye who enter.
:::

::: exi
The Self, smaller than small, greater than great, is hidden in the heart of every creature.
:::

:::: exi
::: gl
| n=an apedani mehuni essandu
| CONN=him that.DAT.SG time.DAT.SG eat.they.shall
| 'They shall celebrate him on that date.'
:::
::::
:::::
\end{verbatim}

Rendered as:

\pex
\a Abandon all hope, ye who enter.

\a The Self, smaller than small, greater than great, is hidden in the
heart of every creature.

\a \begingl
\gla n=an apedani mehuni essandu //
\glb CONN=him that.DAT.SG time.DAT.SG eat.they.shall //
\glft ‘They shall celebrate him on that date.’ //
\endgl

\xe




\end{document}
